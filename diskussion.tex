\section{Diskussion}
\label{sec:Diskussion}

%Die prozentualen Fehler $f$ wurden mithilfe der Formel
%\begin{equation*}
%    f = \frac{\symup{\Delta}W}{W_{\symup{erm}}} 
%\end{equation*}
%bestimmt. Dabei ist $\symup{\Delta}W$ der ermittelte Fehler zum Wert $W_{\symup{erm}}$. 
%Für die Berechnung der Abweichungen $u$ wurde die Formel
%\begin{equation*}
%    u = \biggl| 1-\left(\frac{W_{\symup{erm}}}{W_{\symup{theo}}}\right) \biggr|
%\end{equation*}
%verwendet, wobei $W_{\symup{theo}}$ der Theoriewert ist.

Die großen Schwankungen vor Allem in den höheren Temperaturen im Graphen der Abbildung \ref{fig:cv},
lassen größere Ungenauigkeiten bei der Bestimmung der Molwärme vermuten. Da diese vor allem kurzzeitig 
für die höheren Temperaturen auftreten, könnte es an einem Mangel von flüssigem Stickstoff begründet liegen, 
der ungefähr zu diesen Temperaturen bemerkt wurde und anschließend behoben wurde. Die kleineren 
Schwankungen in den tieferen Temperaturen bedeuten etwas genauere Werte für den Bereich, der 
im Verlauf des Protokolls näher untersucht wurde. Die aus den Werten der tieferen Temperaturen 
berechnete Debye-Temperatur beträgt $\Theta_{\si{D}}=(311,2 \pm 5,5) \, \si{K}$. Der prozentuale Fehler $f$ der bestimmten 
Debye-Temperatur ist also
\begin{equation*}
    f = \frac{\overline{\Theta_{\si{D}}}}{\symup{\Delta}\Theta_{\si{D}}} = 1,8 \%,
\end{equation*}
dies ist ein relativ kleiner Fehler und bestätigt die einheitlicheren Messergebisse bei den tieferen Temperaturen, 
die zur Bestimmung des Wertes verwendet wurden. Also alle Temperaturen unter $170 \, \si{K}$.

Die Abweichung $u$ zum berechneten Theoriewert $\Theta_{\si{D,theo}} = \SI{332,19}{\kelvin}$ sind jedoch 
größer: $u = 6,3 \%$. Eine Abweichung in diesem Bereich ist durch Messungenauigkeiten zu erklären. 
Mögliche Fehlerquellen der Messung sind die Umrechung des Widerstandswertes in die Temperatur und die 
Messungenauigkeit der Messgeräte (sowohl für die Widerstandsmessung, als auch für die Messung der 
Heizspannung und dem Heizstrom). Vor Allem bei der Messung der Heizspannung, die einmal direkt durch 
das Spannungsgerät und einmal durch ein äußeres Messgerät bestimmt wurde, zeigte sich, dass das 
Spannungsgerät keine guten 
Werte für die Spannung lieferte. Die Stromstärke wurde jedoch einfach von der Messanzeige des Spannungsgeräts 
übernommen und nicht weiter überprüft. Weitere Fehler entstanden dadurch, dass die Zeit händisch gestoppt 
wurde, wenn das Ohmmeter den gewünschten Widerstand erreicht hatte. Dabei fließt die Varianz der Reaktionszeit 
mit ein. 

Vor Allem wurde vernachlässigt, dass die Energie aus der Heizspannung und dem Heizstrom gegebenenfalls 
nicht vollständig in Wärmeenergie umgewandelt wurde. Auch ist eine vollständige Wärmeisolation 
eines realen Versuchs nicht möglich. 

Alles in allem ist trotz vieler möglicher Fehlerquellen der Fehler der berechneten Debye-Temperatur recht klein, 
und die Abweichung zum Theoriewert ist auch in einem vertretbaren Bereich. 

